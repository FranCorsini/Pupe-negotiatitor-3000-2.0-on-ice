\documentclass[a4,11pt]{scrartcl} 

\title{Artificial Intelligence Techniques}
\subtitle{Negotiation Agent Design}
\author{\emph{Group 4}\\
\begin{tabular}{ll}
\texttt{4004868}&Tung Phan\\
\texttt{4409159}&Francesco Corsini\\
\texttt{1369326}&Dirk Meijer
\end{tabular}} 

\usepackage[margin=1.5in]{geometry}
\usepackage{hyperref}
\usepackage{graphicx}
\usepackage{amsmath}
\usepackage[sfdefault]{cabin}

\setlength{\parindent}{0pt}
\DeclareTextFontCommand{\emph}{\bf}

\let\tempone\itemize
\let\temptwo\enditemize
\renewenvironment{itemize}{\tempone\addtolength{\itemsep}{-0.5\baselineskip}}{\temptwo}

\begin{document}
\maketitle

\null\vfill
\tableofcontents
\pagebreak

\section{Introduction}

    Negotiation is a complex problem and humans are often not the best 
    negotiators. Emotion and the limited processing capabilities of the 
    human brain can prevent us from getting the best results in 
    negotiations. This makes it an interesting area for AIs. A good 
    negotiation agent can aid humans in negotiation, since they are not 
    limited in the same way humans are.

    The first step is knowing your own \emph{utility}, a quantization 
    of your preferences within the negotiation domain. This allows us 
    to make offers that are agreeable to ourselves and also inspect 
    offers made by other agents, to base our decision of rejection or 
    acception of said offer on. With just this information, it is 
    possible to create a functional agent, although a rather simple 
    one. Such an agent would only make and accept bids that are 
    agreeable to itself. A major issue with this approach is that we 
    don't know in what direction to continue the negotiation, since we 
    only know our own preferences. This means that this strategy might 
    never converge to a solution, depending on the parameters of the 
    scenario.

    A good second step, would be to observe the other agents' behavior, 
    to work out what their utilities are. This allows us to find the 
    solution that gives us the highest utility, \emph{within} the 
    solutions we expect to be acceptable to the other parties. To this 
    end we can think of many different ways to estimate whether a bid 
    will be agreeable to another party, and many different ways to 
    generate bids. \\

    \noindent The assignment was twofold. Firstly we had to define a 
    negotiation domain for three parties, in which different degrees of 
    conflict can exist. The degrees of conflict were specified as 
    collaborative, moderate and competitive. This will be discussed in 
    section \ref{domain}.

    Secondly we were asked to program a negotiation agent in the Java 
    programming language and with the {\sc Genius} negotiation 
    environment. The agent has to work regardless of the used scenario 
    and has to incorporate the preferences of other agents in its 
    decision making process. This will be discussed in section 
    \ref{agent}.


\section{Domain}\label{domain}

    The domain had to exist of multiple issues, each with discrete values.
    Within the domain, we have created three different scenarios that
    correspond to different degrees of conflict.
    
    \paragraph{Background:}
    A specific land zone has turned out to be a perfect place to build 
    a new neighborhood. Up until now it has been used by the farmer to 
    make his cattle go around. The owner of the land is the 
    municipality.

    \paragraph{Parties:}
    \begin{itemize}
        \item Farmer
        \item Construction Company
        \item Municipality
    \end{itemize}

    \paragraph{Issues:}
    \begin{itemize}
        \item Segmentation of the land
        \begin{itemize}
            \item S1: Split 33\%, 33\%, 33\%
            \item S2: 100\% to Farmer
            \item S3: 100\% to Construction Company
            \item S4: 100\% to Municipality
            \item S5: 50\% to Farmer, 50\% to Construction Company
            \item S6: 50\% to Farmer, 50\% to Municipality
            \item S7: 50\% to Construction Company, 50\% to Municipality
        \end{itemize}
        \item Building a water canal
        \begin{itemize}
            \item W1: Big canal
            \item W2: Medium-sized canal
            \item W3: Small canal
            \item W4: No canal
        \end{itemize}
        \item Part of the land reserved for a park
        \begin{itemize}
            \item P1: Big park
            \item P2: Medium-size park
            \item P3: Small park
            \item P4: No park
        \end{itemize}
        \item Building functionality
        \begin{itemize}
            \item F1: Factories
            \item F2: Housing
            \item F3: Shops
            \item F4: Farms
        \end{itemize}
    \end{itemize}
    
    \paragraph{Scenario 1: Competitive} 
    
    The municipality is selling the land. Both the farmer and the 
    construction company want to buy it. But the municipality still 
    wants to retain a zone for welfare structures.
    
    It's going to be difficult for the parties to find an outcome that
    is agreeable to everyone.
    
    \begin{center}
    \begin{tabular}{|l|l|l|l|}
        \hline{}
        {\bf Party}&{\bf Issue}&{\bf Preference}&{\bf Weight}\\
        \hline\hline
        Farmer & Segmentation & S2\textgreater S5=S6\textgreater S1\textgreater S3=S4=S7 & 0.4\\
        \cline{2-4}&Water Canal & W2\textgreater W3\textgreater W1\textgreater W4&0.1\\
        \cline{2-4}&Park& Don't Care & 0.0\\
        \cline{2-4}&Functionality&F4\textgreater F2\textgreater F3\textgreater F1& 0.5\\
        \hline\hline{}
        Construction Company & Segmentation & S3\textgreater S5=S7\textgreater S1\textgreater S2=S4=S6 & 0.4\\
        \cline{2-4}&Water Canal & W2\textgreater W3\textgreater W4\textgreater W1&0.2\\
        \cline{2-4}&Park& P4\textgreater P3\textgreater P2\textgreater P1 & 0.1\\
        \cline{2-4}&Functionality&F3\textgreater F2\textgreater F1\textgreater F4& 0.3\\
        \hline\hline{}
        Municipality & Segmentation & S1\textgreater S5=S6=S7\textgreater S2=S3=S4 & 0.1\\
        \cline{2-4}&Water Canal & W2\textgreater W1\textgreater W3\textgreater W4&0.3\\
        \cline{2-4}&Park& P1\textgreater P2\textgreater P3\textgreater P4 & 0.3\\
        \cline{2-4}&Functionality&F1\textgreater F4\textgreater F3\textgreater F2& 0.2\\
        \hline
    \end{tabular}
    \end{center}

    \paragraph{Scenario 2: Moderate Conflict}
    
    The municipality is selling the land. Both the farmer and the 
    construction company want to buy it. The municipality wants to
    support independent farmers, so its preference is more in alignment
    with the farmer than with the construction company.
    
    This basically reduces to a two-party negotiation, since the 
    preferences for the farmer and the municipality are almost 
    perfectly aligned.
    
    \begin{center}
    \begin{tabular}{|l|l|l|l|}
        \hline{}
        {\bf Party}&{\bf Issue}&{\bf Preference}&{\bf Weight}\\
        \hline\hline
        Farmer & Segmentation & S2\textgreater S1\textgreater S5=S6\textgreater S3=S4=S7 & 0.4\\
        \cline{2-4}&Water Canal & W2\textgreater W3\textgreater W1\textgreater W4&0.2\\
        \cline{2-4}&Park& Don't Care & 0.0\\
        \cline{2-4}&Functionality&F4\textgreater F2\textgreater F3\textgreater F1& 0.4\\
        \hline\hline{}
        Construction Company & Segmentation & S3\textgreater S5=S7\textgreater S1\textgreater S2=S4=S6 & 0.3\\
        \cline{2-4}&Water Canal & W4\textgreater W3\textgreater W2\textgreater W1&0.1\\
        \cline{2-4}&Park& P4\textgreater P3\textgreater P2\textgreater P1 & 0.1\\
        \cline{2-4}&Functionality&F3\textgreater F2\textgreater F1\textgreater F4& 0.5\\
        \hline\hline{}
        Municipality & Segmentation & S2\textgreater S1\textgreater S6\textgreater S5\textgreater S3=S4=S7 & 0.2\\
        \cline{2-4}&Water Canal & W1\textgreater W2\textgreater W3\textgreater W4&0.2\\
        \cline{2-4}&Park& P1\textgreater P2\textgreater P3\textgreater P4 & 0.2\\
        \cline{2-4}&Functionality&F4\textgreater F3\textgreater F1\textgreater F2& 0.4\\
        \hline
    \end{tabular}
    \end{center}
    
    \paragraph{Scenario 3: Collaborative}
    
    The municipality is selling the land, they don't really care what
    happens to it, as long as they no longer have to maintain it. The
    farmer wants to buy the land and build a canal, to irrigate his
    other land. The construction company doesn't necessarily want to
    buy the land, but they don't want it to turn into a big park, and
    they want to build houses, shops or factories on the land.
    
    Since the three parties all care about very different things, they
    should be able to find an outcome for which everyone has a high
    utility.
    
    \begin{center}
    \begin{tabular}{|l|l|l|l|}
        \hline{}
        {\bf Party}&{\bf Issue}&{\bf Preference}&{\bf Weight}\\
        \hline\hline
        Farmer & Segmentation & S2\textgreater S5=S6\textgreater S1\textgreater S3=S4=S7 & 0.2\\
        \cline{2-4}&Water Canal & W1\textgreater W2\textgreater W3\textgreater W4 &0.8\\
        \cline{2-4}&Park& Don't Care & 0.0\\
        \cline{2-4}&Functionality& Don't Care & 0.0\\
        \hline\hline{}
        Construction Company & Segmentation & Don't Care & 0.0\\
        \cline{2-4}&Water Canal & Don't Care &0.0\\
        \cline{2-4}&Park& P4\textgreater P3\textgreater P2\textgreater P1 & 0.3\\
        \cline{2-4}&Functionality& F1=F2=F3\textgreater F4 & 0.7\\
        \hline\hline{}
        Municipality & Segmentation & S2=S3=S5\textgreater S1\textgreater S6=S7\textgreater S4 & 1.0\\
        \cline{2-4}&Water Canal & Don't Care &0.0\\
        \cline{2-4}&Park& Don't Care & 0.0\\
        \cline{2-4}&Functionality& Don't Care & 0.0\\
        \hline
    \end{tabular}
    \end{center}
    


\section{Agent Design}\label{agent}

\subsection{Strategy}

\subsection{Implementation}



\section{Test Results}



\section{Conclusions and Discussion}



\end{document}
