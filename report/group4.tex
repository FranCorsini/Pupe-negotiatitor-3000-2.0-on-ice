\documentclass[a4,10pt]{scrartcl} 
\title{Artificial Intelligence Techniques}
\subtitle{Negotiation Agent Design in Java}
\author{Group 4\\Francesco Corsini - 4409159\\Tung Phan - xxxxxxx\\Dirk Meijer - 1369326} 
\usepackage[margin=2in]{geometry}
\usepackage{hyperref}
\usepackage{graphicx}
\usepackage{amsmath}
\usepackage[sfdefault]{cabin}


\setlength{\parindent}{0pt}


\begin{document}
\maketitle

\null\vfill
\tableofcontents
\pagebreak

\section{Introduction}

Humans are often not the best negotiators. Emotion and the limited 
processing power of the human brain can prevent us from getting the 
best results in negotiations. This makes it an interesting area for 
AIs. A good negotiation agent can aid humans in negotiation, since they 
are not limited in the same way humans are.
Negotiation is a complex problem. The first step is knowing your own 
\emph{utility}, a quantization of your preferences within the 
negotiation domain. This allows us to make offers that are agreeable to
ourselves and also inspect offers made by other agents, to base our
decision of rejection or acception of said offer on.

With just this information, it is possible to create a functional agent,
although a rather simple one. Such an agent would only make and accept
bids that are agreeable to itself. A major issue with this approach is
that we don't know in what direction to continue the negotiation, since
we only know our own preferences.

\section{Domain}

\section{Agent Design}
\subsection{Strategy}
\subsection{Implementation}

\section{Test Results}

\section{Conclusions and Discussion}

\end{document}
